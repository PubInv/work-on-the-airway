%% Copyright (C) 2020 by
%%   Robert L. Read <read.robert@gmail.com>, Megan Cadena <megancad@gmail.com>,
%%   Juan E. Villacres Perez <jvillacres@utexas.edu>
%% Licensed under Creative Commons Attribution-ShareAlike 4.0 (CC BY-SA 4.0)
%% https://creativecommons.org/licenses/by-sa/4.0/


\documentclass{article}
\usepackage{hyperref}
\usepackage{amsmath}
\usepackage{amssymb}
\usepackage{mathtools}
\usepackage{draftwatermark}


\SetWatermarkText{DRAFT}
\SetWatermarkScale{6}
\SetWatermarkLightness{0.95}

\title{Work-on-the-Airway: A Useful Concept for Pandemic Ventilators}

\author{Robert L. Read
  \thanks{read.robert@gmail.com}
  email: \href{mailto:read.robert@gmail.com}{read.robert@gmail.com}\\
Megan Cadena
  \thanks{megancad@gmail.com}
  email: \href{mailto:megancad@gmail.com}{megancad@gmail.com}\\
  Juan E Villacres-Perez
  \thanks{jvillacres@utexas.edu}
  email: \href{mailto:jvillacres@utexas.edu}{jvillacres@utexas.edu}
  }


\begin{document}

\maketitle
\begin{abstract}
  Mechanical ventilation must do work on the airway in order to inflate the lungs.
  Considering the work done on the airway may have several uses:
  \begin{itemize}
  \item Computing the maximum required work on the airway in
    a clinical situation provides engineers a minimum power output requirment
    by an air drive mechanism.
  \item Work-on-the-airway is independent of the means of air production,
    whether by fan, blower, pump, piston, bag-squeezer, or bellows. It therefore
    may serve as a unifying means of controlling air production to meet
    a clinical goal independent of the means of production.
  \item Work-on-the-airway may be used to compute the work of breathing
    if the work done during expiration can be quantified. This may be
    a practical means of offerint work of breathing as a clinical metric.
    \end{itemize}

\end{abstract}


\section{Introduction}

Let us define the term {\em air drive} to mean the mechanism that
produces air and air/oxygen/medical gas mixtures in a mechanical
ventilation system. Because of the
COVID-19 pandemic, many humanitarian engineering teams have
experimented with squeezing inexpensive Bag Mask Valves (BMVs), or
{\em bag squeezers}.
Other mechanisms include pistons, bellows,
positive displacement pumps, which tend to produce a fixed volume against
a variable pressure.
Still other mechanisms such as velocity pumps,
fans and blowers tend to produce a fixed
pressure against a variable back-pressure leading to the injection
of a variable volume. One goal in this paper is to unify these
two very different mechanisms.

MIT has presented a useful computation of the work that must be done
on the airway for maximum patient need, from which they conclude a
mininum power requirement for an air drive for
mechanical ventilation\cite{mitpowercalculation}. Expanding on
this work is a second goal of this paper.


\section{Physical Preliminaries}

Roughly speaking, volume times pressure is work.
To inject an infinitessimal amount of air in
to any air vessel or across any air threshold,
the work is the product of the pressure in at the threshold and the
volume injected.
A threshold into a vessel of fixed size is easy to analyze.
A vessel such as a balloon whose volume is dependent on internal pressure
is slightly more complicated.
A rubber balloon has a {\em compliance} which is defined to be
the change in volume with a change in pressure.

A human lung system is even more
complicated, because it has has
both static and dynamic compliance \url{https://en.wikipedia.org/wiki/Lung_compliance}.

Nonetheless, if we use a simplified model, work done over time
on an airway is the integral over time of injected volume
multipied by pressure, where both injected volume and
pressure are a function of time.

However, by considering maximum needs, the problem is simplified.

\section{Ventilaiton Modes}

Mechanical ventilators offer a number of control modes, the two
simplest for invasive ventilaiton
being pressure-control mode and volume-control mode.
They patient may fight against the action of the ventilator,
called dys-syncrhony. However, if we disregard this clinically
important problem, the modes are simple.

Pressure control mode create inspiration by providing air
at an approximately fixed pressure for a fixed period of time.
Volume control mode pushes air at a potentially variable but
limited to some maximum pressure into the airway until a volume
is acheived.

Volume control mode is easy to acheive with a piston which is
powerful enough: use the piston to push the desired volume of
air out of a cylinder and into the airway. (A weak piston might
not be able to do this.) In so doing we may control the speed
of this push which will somewhat control the pressure.
A positive-displacement pump with a small chamber may be pumped
many times to achieve the desired volume; in this say a
piston and positive-displacement pump are similar.



\section{Work-on-the-Airway as a Ventilation Mode}

We speculate that it might be clinically valuable to
define a ventilation mode which control the work-on-the-airway
done in a given insipriaiton.


\bibliographystyle{acm}

\bibliography{work-on-the-airway}


\end{document}
