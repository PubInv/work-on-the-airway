%% Copyright (C) 2020 by
%%   Robert L. Read <read.robert@gmail.com>, Megan Cadena <megancad@gmail.com>,
%%   Juan E. Villacres Perez <jvillacres@utexas.edu>
%% Licensed under Creative Commons Attribution-ShareAlike 4.0 (CC BY-SA 4.0)
%% https://creativecommons.org/licenses/by-sa/4.0/


\documentclass{article}
\usepackage{hyperref}
\usepackage{amsmath}
\usepackage{amssymb}
\usepackage{mathtools}
\usepackage{draftwatermark}


\SetWatermarkText{DRAFT}
\SetWatermarkScale{6}
\SetWatermarkLightness{0.95}

\title{Power-On-The-Airway: A Useful Concept for Pandemic Ventilators}

\author{Robert L. Read
  \thanks{read.robert@gmail.com}
  email: \href{mailto:read.robert@gmail.com}{read.robert@gmail.com}\\
Megan Cadena
  \thanks{megancad@gmail.com}
  email: \href{mailto:megancad@gmail.com}{megancad@gmail.com}\\
  Juan E Villacres-Perez
  \thanks{jvillacres@utexas.edu}
  email: \href{mailto:jvillacres@utexas.edu}{jvillacres@utexas.edu}
  }


\begin{document}

\maketitle
\begin{abstract}
  Mechanical ventilation must do work on the airway in order to inflate the lungs.
  Considering the power done on the airway may have several uses:
  \begin{itemize}
  \item Computing the maximum required power on the airway in
    a clinical situation provides engineers a minimum power output requirment
    by an air drive mechanism.
  \item Power-on-the-airway is independent of the means of air production,
    whether by fan, blower, pump, piston, bag-squeezer, or bellows. It therefore
    may serve as a unifying means of controlling air production to meet
    a clinical goal independent of the means of production.
  \end{itemize}
  By specifying an {\em air drive} as a modular component in a ventilator
  system that produces air by doing work on the airway according to a
  standardized specification and protocol, it may be possible to
  separate the concern of air prodution from other concerns of building
  ventilators to address the COVID-19 pandemic. This approach provides
  supply chain resillience by allowing air drives to be an interachable part with
  no need for extensive redesign, testing, and certification.
\end{abstract}


\section{Introduction}

Let us define the term {\em air drive} to mean the mechanism that
produces air and air/oxygen/medical gas mixtures in a mechanical
ventilation system. Because of the
COVID-19 pandemic, many humanitarian engineering teams have
experimented with squeezing inexpensive Bag Mask Valves (BMVs), or
{\em bag squeezers}.
Other mechanisms include pistons, bellows,
positive displacement pumps, which tend to produce a fixed volume against
a variable pressure.
Still other mechanisms such as velocity pumps,
fans and blowers tend to produce a fixed
pressure against a variable back-pressure leading to the injection
of a variable volume. One goal in this paper is to unify these
two very different mechanisms.

MIT has presented a useful computation of the work that must be done
on the airway for maximum patient need, from which they conclude a
mininum power requirement for an air drive for
mechanical ventilation\cite{mitpowercalculation}. Expanding on
this work is a second goal of this paper.


\section{Physical Preliminaries}

Roughly speaking, volume times pressure is work.
To inject an infinitessimal amount of air in
to any air vessel or across any air threshold,
the work is the product of the pressure in at the threshold and the
volume injected.
A threshold into a vessel of fixed size is easy to analyze.
A vessel such as a balloon whose volume is dependent on internal pressure
is slightly more complicated.
A rubber balloon has a {\em compliance} which is defined to be
the change in volume with a change in pressure.

A human lung system is even more
complicated, because it has has
both static and dynamic compliance \url{https://en.wikipedia.org/wiki/Lung_compliance}.

Nonetheless, if we use a simplified model, work done over time
on an airway is the integral over time of injected volume
multipied by pressure, where both injected volume and
pressure are a function of time.

If pressure in the airway is a constant $p$ pascals, a machine which produces a flow of $f$ cubic meters
per second, the machine is perforning $p \cdot f$ watts on the airway.


\section{Ventilation Modes}

Mechanical ventilators offer a number of control modes, the two
simplest for invasive ventilaiton
being pressure-control mode and volume-control mode.
They patient may fight against the action of the ventilator,
called dys-syncrhony. However, if we disregard this clinically
important problem, the modes are simple.

Pressure control mode create inspiration by providing air
at an approximately fixed pressure for a fixed period of time.
Volume control mode pushes air at a potentially variable but
limited to some maximum pressure into the airway until a volume
is acheived.

Volume control mode is easy to acheive with a piston which is
powerful enough: use the piston to push the desired volume of
air out of a cylinder and into the airway. (A weak piston might
not be able to do this.) In so doing we may control the speed
of this push which will somewhat control the pressure.
A positive-displacement pump with a small chamber may be pumped
many times to achieve the desired volume; in this say a
piston and positive-displacement pump are similar.

\section{Requirements}

``In a normal person, at rest the work of breathing is about 0.35 J/L, and the power of breathing is about 2.4 J/min.''
\url{https://derangedphysiology.com/main/cicm-primary-exam/required-reading/respiratory-system/Chapter%20041/work-breathing-and-its-components#:~:text=Definitions%20of%20work%20and%20power%20of%20breathing&text=Tada.,is%20about%202.4%20J%2Fmin.}
  ``Normal minute ventilation is between 5 and 8 L per minute (Lpm). Tidal volumes of 500 to 600 mL at 12–14 breaths per minute yield minute ventilations between 6.0 and 8.4 L, for example. Minute ventilation can double with light exercise, and it can exceed 40 Lpm with heavy exercise.''
  \url{https://www.acepnow.com/article/avoid-airway-catastrophes-extremes-minute-ventilation/#:~:text=Normal%20minute%20ventilation%20is%20between,40%20Lpm%20with%20heavy%20exercise.}

    Maximum presumed momentary flow rate: 250 lpm (this is flow rate measured by the Sensirion SFM3200). Momentary flow rate is obviously much higher than minute ventilation
    due to time spent in expiration. (250 > 40.)

    Assume a large adult male is in respiratory distress. They may require a minute volume of 40 Lpm. A disease condition might make this less efficient, so let us
    assume 80 Lpm.

    Maximum airway pressure may be assumed to be 40 to 50 cm H2O.

    A very important reference:   \url{http://www.ubccriticalcaremedicine.ca/rotating/material/Lecture_1%20for%20Residents.pdf}
      states:
      ``Lung compliance will change with age, body position, and various pathological
entities. Normal adult lung compliance ranges from 0.1 to 0.4 L/cm H20. Compliance is
measured under static conditions; that is, under conditions of no flow, in order to
eliminate the factors of resistance from the equation.''

``In a spontaneously breathing adult, normal airway resistance is estimated at 2 to 3
cm H2O/L/sec.''

The seminal MIT E-Vent Power calculation: \url{https://e-vent.mit.edu/mechanical/power-calculation/} performs
a pressure-based evaluation, becasue the Ambubag is guaranteed to have a volume greater than tidal volume (800ml) they apply.

However, by considering maximum needs, the problem is simplified. In mechanical ventilation,
we can choose a maximum pressure of 50 cm H2O and maximum momentary flow rate of 250 liters per minute.
(This may be a conservative overestimate of the maximum power ever needed in a medical emergency.)
Converting to SI units, we convert the pressure of 50cmH2O to 4903.32 Pa and 250 lpm to 0.25 l / min = 0.00416666666 cubic meters per second.
The maximum momentary power on the airway required is the product, 20.4291666667 Watts.
(To convert x cmH2O at a flow rate of y lpm, calculate $x \cdot y \cdot 0.00163433333$)
Following the MIT team, we can conclude, for this choice of clinical conditions, any
machine must provide 20.5 Watts of power. We can immediate concolude, for example, that no machine
that operates at 12 V and draws only 1.5 amps can possibly sufficient, even if 100\% efficient.

\section{Practical Purchasable Machines}

Basic components which may be purchased to make an air drive include:
\begin{itemize}
\item Fans,
\item Centrifugal Pumps,
\item Positive Displacement Pumps,
\item AmbuBags,
\item Pistons,
  \item Pressure valves
\end{itemize}

In practice, the datasheets of fans show the flow of air they proceeed agianst a given pressure. At some pressure,
this flow drops to zero. In general, fans are provide high flow but develop low pressure. They are generally unsuitable
for respiration, which requires lower flow and higher pressure.

The world is full of pumps and compressors. Typical compressors fill tanks or tires to at least 35 psi = 2460.74 cm H2O,
about 50 times more than the higest medical breathing pressures of 50 cm H2O. A typical compressor to produce a 250 lpm
flow (10 cubic feet per minute) costs more than \$200 and requires more than 1 horesepower (746 watts).

To be supply-chain resilient, we would prefer to use commonly available parts.
In general, fans produce too much flow and not enough pressure, and pumps produce too much pressure and not enough flow.
This may be one of the reason for preponderance of ``bag squeezer'' designs in pandemic ventilator projects.
Some blowers designed specifically for CPAP machiens have performance better matched to the breathing task.
These may present supply-chain resillience problems.

Note: Aquarium pumps are closer the performance profile needed for this application.

Note the following language from the RespiraWorks team addresses the problem directly:
\url{https://docs.google.com/document/d/1CE33EcGAdlNdnJA9XW9oGD9veOWSuuBIR2MPmKtkjYQ/edit#}

``Our design centers around a low inertia centrifugal blower, currently sourced from CPAP machines. These brushless fans with lifetime lubricated bearings can spin to high speeds very quickly, allowing a fine degree of time-resolved pressure control. At the same time, they can develop pressures well above 40 cmH2O; even accounting for flow losses, our test model exceeds 100 cmH2O.''

``One of our main assumptions is that access to CPAP blowers will be uninterrupted. We believe both that they are sourced in large quantities (the vendors we’ve contacted in China have more than 5,000 in stock), and that they do not present manufacturing difficulties (fundamentally, it is only three injection molded parts and a brushless DC motor).''

One firm, AirFan, \url{http://www.airfan.fr/mfa0300.html}, makes fans for ventilators using a low-inertial centrifugal pump.

A relatively common approach to ventilation is to assume a source of high-pressure air and control the release of this air
into the airway with a control valve. This has even been done with pure fluidic control which has no moving parts
by directing a flowing airstream into or away from the patient.

In such cases the power-on-the-airway is unrelated to power consumed by operation of the valve, and
depends on the pressure and flow from the pressurized source. However, this is irrelevant to the control system.

\section{Building an Air Drive}

If a blower or centrifugal pump is used as the mechanism of an air drive and the speed of the
blower or pump can be controlled by voltage, and air drive could consist of the blower,
a means of digitally controlling voltage or pulse-width-modulation (PWM), an MCU to receive and interpret commands, and
a map of the voltage  or PWM required to produce a given flow/power at all allowable pressures.
Such an air drive does not need a sensor.
It simply a command in terms of watts, calls a subroutine to look up the voltage in a table
or compute it via interpolation or some formula, and outputs the voltage control.

A positive displacement pump would be different. Generally any such pump produces a stable, known volume displacement with each
stroke or rotation. The air drive would take its command in watts, divide the pressure sent by the controller to obtain
the desired flow and then operate itself at a rate necessary to produce the desired flow.

A pressure gating valve would require that the pressure in the tank be known and the behavior of a valve
be completely understand, but in principle it would also compute a desired flow and operate the valve to acheive that
flow. It might, instead, have its own flow sensor and quickly adjust flow rate to the desired value by its own devices.

The whole point of the air drive is that the effect of all these machines will be unimportant or even unobservable
to doctor and patient. It will not matter to them how the work is done.

\section{Power-on-the-airway of a Typical Blower}



\section{Specification of a Power-on-the-airway Air Drive}

Conceptually an air drive is an air-producing device which can be be controlled by specifying two values:
\begin{itemize}
 \item Watts of work to be done on the airway, and
 \item The pressure of the airway.
\end{itemize}

In practice, these two values must be transmitted electronically to the air drive.
Typically this would be done with an MCU controller that supports I2C, SPI, or a serial interface.
The information could be encoded at the byte-level or in a human-readable format like JSON.
The defintion of a protocol to embody this approach is beyond the scope of this paper.

We assume that when an air drive receives a command, it is required to
do whatever is necessary to produce the specified
watts on the airway if the airway is at the specified pressure.
It is to continue doing this until it receives the next command.

An air drive may or may not have its own ability to sense the pressure in the airway
for its own purposes. It is acceptable for an air drive to be a rather unintelligent
machine that was simply calibrated at manufacture time to produce the required
flow against the specified pressure to produce the required watts.

\section{How to Test a  Power-on-the-airway Air Drive}

The performance of an air drive can be measured with the following values:
\begin{itemize}
\item{power producible against maximum pressure}
\item{duty cycle}
\item{power accuracy: 10\%}
\item{command repsonse time: maximum time to accept a new command in ms}
\item{power response time: maximum change in watts/ms per ms}
\item{mean time to failures in thousands of hours}
\end{itemize}


For example, a good air drive for mechanical ventilation would be
able to produces 21 Watts on the airway at any pressure up to 50 cm H2O, operate at a
50\% duty cycle, and have a response time 1 Watt/ms. Such a response time
would allow the air drive, operating against a suitable mechanical system such
as a test lung, to generate 50 cm H2O pressure at 250 lpm in 21 ms, and
to release that pressure to zero when so ordered in 21 ms.
It could reliably perform this work for thousands of hours at a duty cycle of 50\%.

The resonse time is clinically important in order to allow efficient ventilation
at high respiration rates.
We do not yet have a mathematical model to understand the impact of low response time,
but clearly if the air drive takes a long time to start doing work on the airway
and a long time to release it, gas exchange and possibly even maximum tidal volume
will be impaired. [See Schulz and Read, Anesthesia.]

To test an air drive, attach the air drive to a pneumatic cylinder. Install a flow
and pressure sensor between the two. Weight or activate the pneumatic cylinder
to produce a range of pressures from 0 to the specified maximum. Command the air drive
to produce a range of powers up to the maximum specified power. Perform this
from a ``standing start'' or zero power and returning to zero power. The response time
can be computed from the pressure and flow curve. At any point in time, power-on-the-airway
is pressure times flows. Measure that the power-on-the-airway is within the specified
accuracy.

If you don't have a pneumatic cylinder, a test lung can accomplish approximately the
same test because it's small volume will quickly be pressurized. This will require
a bit more study of power-on-the-airway to produce pressure, which will be a function
of the restriction on the test lung, compliance, and total volume. Nonetheless it should
be possible by ramping up power to measure all parameters at all pressures.

\section{Implementing Ventilation Modes With Power-on-the-airway}

In order to accomplish interchangability of air drives within a ventilator without
changing the observable behavior of the ventilator, we imagine a controller which
is sending commands to commands to the air drive.
This controller implements one or more ventilation modes.
The controller represent an algorithm for implementing a ventilation mode
in terms of power-on-the-airway. In almost all cases, the algorithm will use
pressure and possibly flow sensors on the airway that the controller can read.
The air drive may not be able to sense these things itself.

There are a number of ways such an algorithm could be implemented, but
one of the most familiar would be as a PID controller. In the terminology
of such systems, the power-on-the-airway in watts is the {\em control variable},
but the {\em error value} depends on the ventilation mode.

\subsection{Pressure Control Mode}

Pressure Controlled Ventilation (PCV) is perhaps the most basic.
In this mode the error value of the PID controller would be the difference
between the desired PIP and the airway pressure during the inspiration phase,
which is a fixed time, and the difference between the desired PEEP and the
airway pressure during the fixed expiration period. A controller with a single
airway pressure sensor can implement this mode.

\subsection{Volume Control Mode}

Volume Controlled Ventilation (VCV) may assign a fixed flow rate to
performed on the airway until a tidal volume is acheieved.
(Generally there remains
a maximum pressure, either implemented mechanically with a pop-off valve,
or electronically.)
Since
flow rate and tidal volume are fixed, the time of inspiration
is calculate by tidal volume divided by flow rate.
Such a mode can be implemented in two ways. If the controller
has a flow sensor, the flow in the airway subtracted from the
prescribed flow can be the error value. However, interestingly, if
we have an airdrive, the controller could implement volume control mode with a single
pressue sensor, by multiplying the desired flow rate times the current pressure
to produce the watts to command the air drive to produce. If the air drive does its
job, the flow will be accurate to with the specified accuracy.

Note that in either case the simple and well-established PID controller
approach can be used.

\subsection{Power-on-the-airway as a clinical measure}

The integration of power over time is work. Although a patient's
lung restriction and compliance may change over time, any inspiration
provided by a power-on-the-airway drive automatically provides
the power of breathing if you simply sum up the watts commanded
and divided by time in each time interval between commands.
A controller using a power-on-the-airway drive thus almost
automaticaly computes the inspiratory work of breathing.

We speculate that it might be clinically valuable to
define a ventilation mode which controls the power-on-the-airway
done in a given insipriation.


\bibliographystyle{acm}

\bibliography{power-on-the-airway}


\end{document}
