%% Copyright (C) 2020 by
%%   Robert L. Read <read.robert@gmail.com>, Megan Cadena <megancad@gmail.com>,
%%   Juan E. Villacres Perez <jvillacres@utexas.edu>
%% Licensed under Creative Commons Attribution-ShareAlike 4.0 (CC BY-SA 4.0)
%% https://creativecommons.org/licenses/by-sa/4.0/


\documentclass{article}
\usepackage{hyperref}
\usepackage{amsmath}
\usepackage{amssymb}
\usepackage{mathtools}
\usepackage{draftwatermark}


\SetWatermarkText{DRAFT}
\SetWatermarkScale{6}
\SetWatermarkLightness{0.95}

\title{Work-on-the-Airway: A Useful Concept for Pandemic Ventilators}

\author{Robert L. Read
  \thanks{read.robert@gmail.com}
  email: \href{mailto:read.robert@gmail.com}{read.robert@gmail.com}\\
Megan Cadena
  \thanks{megancad@gmail.com}
  email: \href{mailto:megancad@gmail.com}{megancad@gmail.com}\\
  Juan E Villacres-Perez
  \thanks{jvillacres@utexas.edu}
  email: \href{mailto:jvillacres@utexas.edu}{jvillacres@utexas.edu}
  }


\begin{document}

\maketitle
\begin{abstract}
  Mechanical ventilation must do work on the airway in order to inflate the lungs.
  Considering the work done on the airway may have several uses:
  \begin{itemize}
  \item Computing the maximum required work on the airway in
    a clinical situation provides engineers a minimum power output requirment
    by an air drive mechanism.
  \item Work-on-the-airway is independent of the means of air production,
    whether by fan, blower, pump, piston, bag-squeezer, or bellows. It therefore
    may serve as a unifying means of controlling air production to meet
    a clinical goal independent of the means of production.
    \end{itemize}

\end{abstract}


\section{Introduction}

Let us define the term {\em air drive} to mean the mechanism that
produces air and air/oxygen/medical gas mixtures in a mechanical
ventilation system. Because of the
COVID-19 pandemic, many humanitarian engineering teams have
experimented with squeezing inexpensive Bag Mask Valves (BMVs), or
{\em bag squeezers}.
Other mechanisms include pistons, bellows,
positive displacement pumps, which tend to produce a fixed volume against
a variable pressure.
Still other mechanisms such as velocity pumps,
fans and blowers tend to produce a fixed
pressure against a variable back-pressure leading to the injection
of a variable volume. One goal in this paper is to unify these
two very different mechanisms.

MIT has presented a useful computation of the work that must be done
on the airway for maximum patient need, from which they conclude a
mininum power requirement for an air drive for
mechanical ventilation\cite{mitpowercalculation}. Expanding on
this work is a second goal of this paper.


\section{Physical Preliminaries}

Roughly speaking, volume times pressure is work.
To inject an infinitessimal amount of air in
to any air vessel or across any air threshold,
the work is the product of the pressure in at the threshold and the
volume injected.
A threshold into a vessel of fixed size is easy to analyze.
A vessel such as a balloon whose volume is dependent on internal pressure
is slightly more complicated.
A rubber balloon has a {\em compliance} which is defined to be
the change in volume with a change in pressure.

A human lung system is even more
complicated, because it has has
both static and dynamic compliance \url{https://en.wikipedia.org/wiki/Lung_compliance}.

Nonetheless, if we use a simplified model, work done over time
on an airway is the integral over time of injected volume
multipied by pressure, where both injected volume and
pressure are a function of time.

However, by considering maximum needs, the problem is simplified.

\section{Ventilation Modes}

Mechanical ventilators offer a number of control modes, the two
simplest for invasive ventilaiton
being pressure-control mode and volume-control mode.
They patient may fight against the action of the ventilator,
called dys-syncrhony. However, if we disregard this clinically
important problem, the modes are simple.

Pressure control mode create inspiration by providing air
at an approximately fixed pressure for a fixed period of time.
Volume control mode pushes air at a potentially variable but
limited to some maximum pressure into the airway until a volume
is acheived.

Volume control mode is easy to acheive with a piston which is
powerful enough: use the piston to push the desired volume of
air out of a cylinder and into the airway. (A weak piston might
not be able to do this.) In so doing we may control the speed
of this push which will somewhat control the pressure.
A positive-displacement pump with a small chamber may be pumped
many times to achieve the desired volume; in this say a
piston and positive-displacement pump are similar.

\section{Requirements}

``In a normal person, at rest the work of breathing is about 0.35 J/L, and the power of breathing is about 2.4 J/min.''
\url{https://derangedphysiology.com/main/cicm-primary-exam/required-reading/respiratory-system/Chapter%20041/work-breathing-and-its-components#:~:text=Definitions%20of%20work%20and%20power%20of%20breathing&text=Tada.,is%20about%202.4%20J%2Fmin.}
  ``Normal minute ventilation is between 5 and 8 L per minute (Lpm). Tidal volumes of 500 to 600 mL at 12–14 breaths per minute yield minute ventilations between 6.0 and 8.4 L, for example. Minute ventilation can double with light exercise, and it can exceed 40 Lpm with heavy exercise.''
  \url{https://www.acepnow.com/article/avoid-airway-catastrophes-extremes-minute-ventilation/#:~:text=Normal%20minute%20ventilation%20is%20between,40%20Lpm%20with%20heavy%20exercise.}

    Maximum presumed momentary flow rate: 250 lpm (this is flow rate measured by the Sensirion SFM3200). Momentary flow rate is obviously much higher than minute ventilation
    due to time spent in expiration. (250 > 40.)

    Assume a large adult male is in respiratory distress. They may require a minute volume of 40 Lpm. A disease condition might make this less efficient, so let us
    assume 80 Lpm.

    Maximum airway pressure may be assumed to be 40 to 50 cm H2O.

    A very important reference:   \url{http://www.ubccriticalcaremedicine.ca/rotating/material/Lecture_1%20for%20Residents.pdf}
      states:
      ``Lung compliance will change with age, body position, and various pathological
entities. Normal adult lung compliance ranges from 0.1 to 0.4 L/cm H20. Compliance is
measured under static conditions; that is, under conditions of no flow, in order to
eliminate the factors of resistance from the equation.''

``In a spontaneously breathing adult, normal airway resistance is estimated at 2 to 3
cm H2O/L/sec.''

The seminal MIT E-Vent Power calculation: \url{https://e-vent.mit.edu/mechanical/power-calculation/} performs
a pressure-based evaluation, becasue the Ambubag is guaranteed to have a volume greater than tidal volume (800ml) they apply.

\section{Practical Purchasable Machines}

Basic components which may be purchased to make an air drive include:
\begin{itemize}
\item Fans,
\item Centrifugal Pumps,
\item Positive Displacement Pumps,
\item AmbuBags
\item Pistons
\end{itemize}

In practice, the datasheets of fans show the flow of air they proceeed agianst a given pressure. At some pressure,
this flow drops to zero. In general, fans are provide high flow but develop low pressure. They are generally unsuitable
for respiration, which requires lower flow and higher pressure.

The world is full of pumps and compressors. Typical compressors fill tanks or tires to at least 35 psi = 2460.74 cm H2O,
about 50 times more than the higest medical breathing pressures of 50 cm H2O. A typical compressor to produce a 250 lpm
flow (10 cubic feet per minute) costs more than \$200 and requires more than 1 horesepower (746 watts).

This may be one of the reason for preponderance of ``bag squeezer'' designs in pandemic ventilator projects.

Some blowers designed specifically for CPAP machiens have performance better matched to the breathing task.
These may present supply-chain resillience problems.

Note: Aquarium pumps are closer the performance profile needed for this application.

Note the following language from the RespiraWorks team addresses the problem directly:
\url{https://docs.google.com/document/d/1CE33EcGAdlNdnJA9XW9oGD9veOWSuuBIR2MPmKtkjYQ/edit#}

``Our design centers around a low inertia centrifugal blower, currently sourced from CPAP machines. These brushless fans with lifetime lubricated bearings can spin to high speeds very quickly, allowing a fine degree of time-resolved pressure control. At the same time, they can develop pressures well above 40 cmH2O; even accounting for flow losses, our test model exceeds 100 cmH2O.''

``One of our main assumptions is that access to CPAP blowers will be uninterrupted. We believe both that they are sourced in large quantities (the vendors we’ve contacted in China have more than 5,000 in stock), and that they do not present manufacturing difficulties (fundamentally, it is only three injection molded parts and a brushless DC motor).''




\section{Work-on-the-Airway of a Typical Blower}

Here take the RespiraWorks blower and do an analysis of it.


\section{Specification of a Work-on-the-Airway Air Drive}

\begin{itemize}
\item{work against maximum pressure}
\item{duty cycle}
\item{accuracy: 10\%}
\item{response time: minimum change in work/ms per ms}
\end{itemize}

Note response time should relate to Erich Schultz's concepts of
TAIP and TRIP: Time to Acheive Inspiratory Pressure and
Time to Release Inspiratory Pressure.

\section{How to Test a  Work-on-the-Airway Air Drive}



\section{Implementing Ventilation Modes With Work-on-the-Airway}

\subsection{Pressure Control Mode}
\subsection{Volume Control Mode}


\subsection{Work-on-the-Airway as a Ventilation Mode}

We speculate that it might be clinically valuable to
define a ventilation mode which control the work-on-the-airway
done in a given insipriaiton.


\bibliographystyle{acm}

\bibliography{work-on-the-airway}


\end{document}
